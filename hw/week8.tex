\documentclass{report}

%%%%%%%%%%%%%%%%%%%%%%%%%%%%%%%%%
% PACKAGE IMPORTS
%%%%%%%%%%%%%%%%%%%%%%%%%%%%%%%%%


\usepackage[tmargin=2cm,rmargin=1in,lmargin=1in,margin=0.85in,bmargin=2cm,footskip=.2in]{geometry}
\usepackage{amsmath,amsfonts,amsthm,amssymb,mathtools}
\usepackage[varbb]{newpxmath}
\usepackage{xfrac}
\usepackage[makeroom]{cancel}
\usepackage{mathtools}
\usepackage{bookmark}
\usepackage{enumitem}
\usepackage{hyperref,theoremref}
\hypersetup{
	pdftitle={Assignment},
	colorlinks=true, linkcolor=doc!90,
	bookmarksnumbered=true,
	bookmarksopen=true
}
\usepackage[most,many,breakable]{tcolorbox}
\usepackage{xcolor}
\usepackage{varwidth}
\usepackage{varwidth}
\usepackage{etoolbox}
%\usepackage{authblk}
\usepackage{nameref}
\usepackage{multicol,array}
\usepackage{tikz-cd}
\usepackage[ruled,vlined,linesnumbered]{algorithm2e}
\usepackage{comment} % enables the use of multi-line comments (\ifx \fi) 
\usepackage{import}
\usepackage{xifthen}
\usepackage{pdfpages}
\usepackage{transparent}

\newcommand\mycommfont[1]{\footnotesize\ttfamily\textcolor{blue}{#1}}
\SetCommentSty{mycommfont}
\newcommand{\incfig}[1]{%
    \def\svgwidth{\columnwidth}
    \import{./figures/}{#1.pdf_tex}
}

\usepackage{tikzsymbols}
\renewcommand\qedsymbol{$\Laughey$}


%\usepackage{import}
%\usepackage{xifthen}
%\usepackage{pdfpages}
%\usepackage{transparent}


%%%%%%%%%%%%%%%%%%%%%%%%%%%%%%
% SELF MADE COLORS
%%%%%%%%%%%%%%%%%%%%%%%%%%%%%%



\definecolor{myg}{RGB}{56, 140, 70}
\definecolor{myb}{RGB}{45, 111, 177}
\definecolor{myr}{RGB}{199, 68, 64}
\definecolor{mytheorembg}{HTML}{F2F2F9}
\definecolor{mytheoremfr}{HTML}{00007B}
\definecolor{mylenmabg}{HTML}{FFFAF8}
\definecolor{mylenmafr}{HTML}{983b0f}
\definecolor{mypropbg}{HTML}{f2fbfc}
\definecolor{mypropfr}{HTML}{191971}
\definecolor{myexamplebg}{HTML}{F2FBF8}
\definecolor{myexamplefr}{HTML}{88D6D1}
\definecolor{myexampleti}{HTML}{2A7F7F}
\definecolor{mydefinitbg}{HTML}{E5E5FF}
\definecolor{mydefinitfr}{HTML}{3F3FA3}
\definecolor{notesgreen}{RGB}{0,162,0}
\definecolor{myp}{RGB}{197, 92, 212}
\definecolor{mygr}{HTML}{2C3338}
\definecolor{myred}{RGB}{127,0,0}
\definecolor{myyellow}{RGB}{169,121,69}
\definecolor{myexercisebg}{HTML}{F2FBF8}
\definecolor{myexercisefg}{HTML}{88D6D1}


%%%%%%%%%%%%%%%%%%%%%%%%%%%%
% TCOLORBOX SETUPS
%%%%%%%%%%%%%%%%%%%%%%%%%%%%

\setlength{\parindent}{1cm}
%================================
% THEOREM BOX
%================================

\tcbuselibrary{theorems,skins,hooks}
\newtcbtheorem[number within=section]{Theorem}{Theorem}
{%
	enhanced,
	breakable,
	colback = mytheorembg,
	frame hidden,
	boxrule = 0sp,
	borderline west = {2pt}{0pt}{mytheoremfr},
	sharp corners,
	detach title,
	before upper = \tcbtitle\par\smallskip,
	coltitle = mytheoremfr,
	fonttitle = \bfseries\sffamily,
	description font = \mdseries,
	separator sign none,
	segmentation style={solid, mytheoremfr},
}
{th}

\tcbuselibrary{theorems,skins,hooks}
\newtcbtheorem[number within=chapter]{theorem}{Theorem}
{%
	enhanced,
	breakable,
	colback = mytheorembg,
	frame hidden,
	boxrule = 0sp,
	borderline west = {2pt}{0pt}{mytheoremfr},
	sharp corners,
	detach title,
	before upper = \tcbtitle\par\smallskip,
	coltitle = mytheoremfr,
	fonttitle = \bfseries\sffamily,
	description font = \mdseries,
	separator sign none,
	segmentation style={solid, mytheoremfr},
}
{th}


\tcbuselibrary{theorems,skins,hooks}
\newtcolorbox{Theoremcon}
{%
	enhanced
	,breakable
	,colback = mytheorembg
	,frame hidden
	,boxrule = 0sp
	,borderline west = {2pt}{0pt}{mytheoremfr}
	,sharp corners
	,description font = \mdseries
	,separator sign none
}

%================================
% Corollery
%================================
\tcbuselibrary{theorems,skins,hooks}
\newtcbtheorem[number within=section]{Corollary}{Corollary}
{%
	enhanced
	,breakable
	,colback = myp!10
	,frame hidden
	,boxrule = 0sp
	,borderline west = {2pt}{0pt}{myp!85!black}
	,sharp corners
	,detach title
	,before upper = \tcbtitle\par\smallskip
	,coltitle = myp!85!black
	,fonttitle = \bfseries\sffamily
	,description font = \mdseries
	,separator sign none
	,segmentation style={solid, myp!85!black}
}
{th}
\tcbuselibrary{theorems,skins,hooks}
\newtcbtheorem[number within=chapter]{corollary}{Corollary}
{%
	enhanced
	,breakable
	,colback = myp!10
	,frame hidden
	,boxrule = 0sp
	,borderline west = {2pt}{0pt}{myp!85!black}
	,sharp corners
	,detach title
	,before upper = \tcbtitle\par\smallskip
	,coltitle = myp!85!black
	,fonttitle = \bfseries\sffamily
	,description font = \mdseries
	,separator sign none
	,segmentation style={solid, myp!85!black}
}
{th}


%================================
% LENMA
%================================

\tcbuselibrary{theorems,skins,hooks}
\newtcbtheorem[number within=section]{Lenma}{Lenma}
{%
	enhanced,
	breakable,
	colback = mylenmabg,
	frame hidden,
	boxrule = 0sp,
	borderline west = {2pt}{0pt}{mylenmafr},
	sharp corners,
	detach title,
	before upper = \tcbtitle\par\smallskip,
	coltitle = mylenmafr,
	fonttitle = \bfseries\sffamily,
	description font = \mdseries,
	separator sign none,
	segmentation style={solid, mylenmafr},
}
{th}

\tcbuselibrary{theorems,skins,hooks}
\newtcbtheorem[number within=chapter]{lenma}{Lenma}
{%
	enhanced,
	breakable,
	colback = mylenmabg,
	frame hidden,
	boxrule = 0sp,
	borderline west = {2pt}{0pt}{mylenmafr},
	sharp corners,
	detach title,
	before upper = \tcbtitle\par\smallskip,
	coltitle = mylenmafr,
	fonttitle = \bfseries\sffamily,
	description font = \mdseries,
	separator sign none,
	segmentation style={solid, mylenmafr},
}
{th}


%================================
% PROPOSITION
%================================

\tcbuselibrary{theorems,skins,hooks}
\newtcbtheorem[number within=section]{Prop}{Proposition}
{%
	enhanced,
	breakable,
	colback = mypropbg,
	frame hidden,
	boxrule = 0sp,
	borderline west = {2pt}{0pt}{mypropfr},
	sharp corners,
	detach title,
	before upper = \tcbtitle\par\smallskip,
	coltitle = mypropfr,
	fonttitle = \bfseries\sffamily,
	description font = \mdseries,
	separator sign none,
	segmentation style={solid, mypropfr},
}
{th}

\tcbuselibrary{theorems,skins,hooks}
\newtcbtheorem[number within=chapter]{prop}{Proposition}
{%
	enhanced,
	breakable,
	colback = mypropbg,
	frame hidden,
	boxrule = 0sp,
	borderline west = {2pt}{0pt}{mypropfr},
	sharp corners,
	detach title,
	before upper = \tcbtitle\par\smallskip,
	coltitle = mypropfr,
	fonttitle = \bfseries\sffamily,
	description font = \mdseries,
	separator sign none,
	segmentation style={solid, mypropfr},
}
{th}


%================================
% CLAIM
%================================

\tcbuselibrary{theorems,skins,hooks}
\newtcbtheorem[number within=section]{claim}{Claim}
{%
	enhanced
	,breakable
	,colback = myg!10
	,frame hidden
	,boxrule = 0sp
	,borderline west = {2pt}{0pt}{myg}
	,sharp corners
	,detach title
	,before upper = \tcbtitle\par\smallskip
	,coltitle = myg!85!black
	,fonttitle = \bfseries\sffamily
	,description font = \mdseries
	,separator sign none
	,segmentation style={solid, myg!85!black}
}
{th}



%================================
% Exercise
%================================

\tcbuselibrary{theorems,skins,hooks}
\newtcbtheorem[number within=section]{Exercise}{Exercise}
{%
	enhanced,
	breakable,
	colback = myexercisebg,
	frame hidden,
	boxrule = 0sp,
	borderline west = {2pt}{0pt}{myexercisefg},
	sharp corners,
	detach title,
	before upper = \tcbtitle\par\smallskip,
	coltitle = myexercisefg,
	fonttitle = \bfseries\sffamily,
	description font = \mdseries,
	separator sign none,
	segmentation style={solid, myexercisefg},
}
{th}

\tcbuselibrary{theorems,skins,hooks}
\newtcbtheorem[number within=chapter]{exercise}{Exercise}
{%
	enhanced,
	breakable,
	colback = myexercisebg,
	frame hidden,
	boxrule = 0sp,
	borderline west = {2pt}{0pt}{myexercisefg},
	sharp corners,
	detach title,
	before upper = \tcbtitle\par\smallskip,
	coltitle = myexercisefg,
	fonttitle = \bfseries\sffamily,
	description font = \mdseries,
	separator sign none,
	segmentation style={solid, myexercisefg},
}
{th}

%================================
% EXAMPLE BOX
%================================

\newtcbtheorem[number within=section]{Example}{Example}
{%
	colback = myexamplebg
	,breakable
	,colframe = myexamplefr
	,coltitle = myexampleti
	,boxrule = 1pt
	,sharp corners
	,detach title
	,before upper=\tcbtitle\par\smallskip
	,fonttitle = \bfseries
	,description font = \mdseries
	,separator sign none
	,description delimiters parenthesis
}
{ex}

\newtcbtheorem[number within=chapter]{example}{Example}
{%
	colback = myexamplebg
	,breakable
	,colframe = myexamplefr
	,coltitle = myexampleti
	,boxrule = 1pt
	,sharp corners
	,detach title
	,before upper=\tcbtitle\par\smallskip
	,fonttitle = \bfseries
	,description font = \mdseries
	,separator sign none
	,description delimiters parenthesis
}
{ex}

%================================
% DEFINITION BOX
%================================

\newtcbtheorem[number within=section]{Definition}{Definition}{enhanced,
	before skip=2mm,after skip=2mm, colback=red!5,colframe=red!80!black,boxrule=0.5mm,
	attach boxed title to top left={xshift=1cm,yshift*=1mm-\tcboxedtitleheight}, varwidth boxed title*=-3cm,
	boxed title style={frame code={
					\path[fill=tcbcolback]
					([yshift=-1mm,xshift=-1mm]frame.north west)
					arc[start angle=0,end angle=180,radius=1mm]
					([yshift=-1mm,xshift=1mm]frame.north east)
					arc[start angle=180,end angle=0,radius=1mm];
					\path[left color=tcbcolback!60!black,right color=tcbcolback!60!black,
						middle color=tcbcolback!80!black]
					([xshift=-2mm]frame.north west) -- ([xshift=2mm]frame.north east)
					[rounded corners=1mm]-- ([xshift=1mm,yshift=-1mm]frame.north east)
					-- (frame.south east) -- (frame.south west)
					-- ([xshift=-1mm,yshift=-1mm]frame.north west)
					[sharp corners]-- cycle;
				},interior engine=empty,
		},
	fonttitle=\bfseries,
	title={#2},#1}{def}
\newtcbtheorem[number within=chapter]{definition}{Definition}{enhanced,
	before skip=2mm,after skip=2mm, colback=red!5,colframe=red!80!black,boxrule=0.5mm,
	attach boxed title to top left={xshift=1cm,yshift*=1mm-\tcboxedtitleheight}, varwidth boxed title*=-3cm,
	boxed title style={frame code={
					\path[fill=tcbcolback]
					([yshift=-1mm,xshift=-1mm]frame.north west)
					arc[start angle=0,end angle=180,radius=1mm]
					([yshift=-1mm,xshift=1mm]frame.north east)
					arc[start angle=180,end angle=0,radius=1mm];
					\path[left color=tcbcolback!60!black,right color=tcbcolback!60!black,
						middle color=tcbcolback!80!black]
					([xshift=-2mm]frame.north west) -- ([xshift=2mm]frame.north east)
					[rounded corners=1mm]-- ([xshift=1mm,yshift=-1mm]frame.north east)
					-- (frame.south east) -- (frame.south west)
					-- ([xshift=-1mm,yshift=-1mm]frame.north west)
					[sharp corners]-- cycle;
				},interior engine=empty,
		},
	fonttitle=\bfseries,
	title={#2},#1}{def}



%================================
% Solution BOX
%================================

\makeatletter
\newtcbtheorem{question}{Question}{enhanced,
	breakable,
	colback=white,
	colframe=myb!80!black,
	attach boxed title to top left={yshift*=-\tcboxedtitleheight},
	fonttitle=\bfseries,
	title={#2},
	boxed title size=title,
	boxed title style={%
			sharp corners,
			rounded corners=northwest,
			colback=tcbcolframe,
			boxrule=0pt,
		},
	underlay boxed title={%
			\path[fill=tcbcolframe] (title.south west)--(title.south east)
			to[out=0, in=180] ([xshift=5mm]title.east)--
			(title.center-|frame.east)
			[rounded corners=\kvtcb@arc] |-
			(frame.north) -| cycle;
		},
	#1
}{def}
\makeatother

%================================
% SOLUTION BOX
%================================

\makeatletter
\newtcolorbox{solution}{enhanced,
	breakable,
	colback=white,
	colframe=myg!80!black,
	attach boxed title to top left={yshift*=-\tcboxedtitleheight},
	title=Solution,
	boxed title size=title,
	boxed title style={%
			sharp corners,
			rounded corners=northwest,
			colback=tcbcolframe,
			boxrule=0pt,
		},
	underlay boxed title={%
			\path[fill=tcbcolframe] (title.south west)--(title.south east)
			to[out=0, in=180] ([xshift=5mm]title.east)--
			(title.center-|frame.east)
			[rounded corners=\kvtcb@arc] |-
			(frame.north) -| cycle;
		},
}
\makeatother

%================================
% Question BOX
%================================

\makeatletter
\newtcbtheorem{qstion}{Question}{enhanced,
	breakable,
	colback=white,
	colframe=mygr,
	attach boxed title to top left={yshift*=-\tcboxedtitleheight},
	fonttitle=\bfseries,
	title={#2},
	boxed title size=title,
	boxed title style={%
			sharp corners,
			rounded corners=northwest,
			colback=tcbcolframe,
			boxrule=0pt,
		},
	underlay boxed title={%
			\path[fill=tcbcolframe] (title.south west)--(title.south east)
			to[out=0, in=180] ([xshift=5mm]title.east)--
			(title.center-|frame.east)
			[rounded corners=\kvtcb@arc] |-
			(frame.north) -| cycle;
		},
	#1
}{def}
\makeatother

\newtcbtheorem[number within=chapter]{wconc}{Wrong Concept}{
	breakable,
	enhanced,
	colback=white,
	colframe=myr,
	arc=0pt,
	outer arc=0pt,
	fonttitle=\bfseries\sffamily\large,
	colbacktitle=myr,
	attach boxed title to top left={},
	boxed title style={
			enhanced,
			skin=enhancedfirst jigsaw,
			arc=3pt,
			bottom=0pt,
			interior style={fill=myr}
		},
	#1
}{def}



%================================
% NOTE BOX
%================================

\usetikzlibrary{arrows,calc,shadows.blur}
\tcbuselibrary{skins}
\newtcolorbox{note}[1][]{%
	enhanced jigsaw,
	colback=gray!20!white,%
	colframe=gray!80!black,
	size=small,
	boxrule=1pt,
	title=\textbf{Note:-},
	halign title=flush center,
	coltitle=black,
	breakable,
	drop shadow=black!50!white,
	attach boxed title to top left={xshift=1cm,yshift=-\tcboxedtitleheight/2,yshifttext=-\tcboxedtitleheight/2},
	minipage boxed title=1.5cm,
	boxed title style={%
			colback=white,
			size=fbox,
			boxrule=1pt,
			boxsep=2pt,
			underlay={%
					\coordinate (dotA) at ($(interior.west) + (-0.5pt,0)$);
					\coordinate (dotB) at ($(interior.east) + (0.5pt,0)$);
					\begin{scope}
						\clip (interior.north west) rectangle ([xshift=3ex]interior.east);
						\filldraw [white, blur shadow={shadow opacity=60, shadow yshift=-.75ex}, rounded corners=2pt] (interior.north west) rectangle (interior.south east);
					\end{scope}
					\begin{scope}[gray!80!black]
						\fill (dotA) circle (2pt);
						\fill (dotB) circle (2pt);
					\end{scope}
				},
		},
	#1,
}

%%%%%%%%%%%%%%%%%%%%%%%%%%%%%%
% SELF MADE COMMANDS
%%%%%%%%%%%%%%%%%%%%%%%%%%%%%%


\newcommand{\thm}[2]{\begin{Theorem}{#1}{}#2\end{Theorem}}
\newcommand{\cor}[2]{\begin{Corollary}{#1}{}#2\end{Corollary}}
\newcommand{\mlenma}[2]{\begin{Lenma}{#1}{}#2\end{Lenma}}
\newcommand{\mprop}[2]{\begin{Prop}{#1}{}#2\end{Prop}}
\newcommand{\clm}[3]{\begin{claim}{#1}{#2}#3\end{claim}}
\newcommand{\wc}[2]{\begin{wconc}{#1}{}\setlength{\parindent}{1cm}#2\end{wconc}}
\newcommand{\thmcon}[1]{\begin{Theoremcon}{#1}\end{Theoremcon}}
\newcommand{\ex}[2]{\begin{Example}{#1}{}#2\end{Example}}
\newcommand{\dfn}[2]{\begin{Definition}[colbacktitle=red!75!black]{#1}{}#2\end{Definition}}
\newcommand{\dfnc}[2]{\begin{definition}[colbacktitle=red!75!black]{#1}{}#2\end{definition}}
\newcommand{\qs}[2]{\begin{question}{#1}{}#2\end{question}}
\newcommand{\pf}[2]{\begin{myproof}[#1]#2\end{myproof}}
\newcommand{\nt}[1]{\begin{note}#1\end{note}}

\newcommand*\circled[1]{\tikz[baseline=(char.base)]{
		\node[shape=circle,draw,inner sep=1pt] (char) {#1};}}
\newcommand\getcurrentref[1]{%
	\ifnumequal{\value{#1}}{0}
	{??}
	{\the\value{#1}}%
}
\newcommand{\getCurrentSectionNumber}{\getcurrentref{section}}
\newenvironment{myproof}[1][\proofname]{%
	\proof[\bfseries #1: ]%
}{\endproof}

\newcommand{\mclm}[2]{\begin{myclaim}[#1]#2\end{myclaim}}
\newenvironment{myclaim}[1][\claimname]{\proof[\bfseries #1: ]}{}

\newcounter{mylabelcounter}

\makeatletter
\newcommand{\setword}[2]{%
	\phantomsection
	#1\def\@currentlabel{\unexpanded{#1}}\label{#2}%
}
\makeatother




\tikzset{
	symbol/.style={
			draw=none,
			every to/.append style={
					edge node={node [sloped, allow upside down, auto=false]{$#1$}}}
		}
}


% deliminators
\DeclarePairedDelimiter{\abs}{\lvert}{\rvert}
\DeclarePairedDelimiter{\norm}{\lVert}{\rVert}

\DeclarePairedDelimiter{\ceil}{\lceil}{\rceil}
\DeclarePairedDelimiter{\floor}{\lfloor}{\rfloor}
\DeclarePairedDelimiter{\round}{\lfloor}{\rceil}

\newsavebox\diffdbox
\newcommand{\slantedromand}{{\mathpalette\makesl{d}}}
\newcommand{\makesl}[2]{%
\begingroup
\sbox{\diffdbox}{$\mathsurround=0pt#1\mathrm{#2}$}%
\pdfsave
\pdfsetmatrix{1 0 0.2 1}%
\rlap{\usebox{\diffdbox}}%
\pdfrestore
\hskip\wd\diffdbox
\endgroup
}
\newcommand{\dd}[1][]{\ensuremath{\mathop{}\!\ifstrempty{#1}{%
\slantedromand\@ifnextchar^{\hspace{0.2ex}}{\hspace{0.1ex}}}%
{\slantedromand\hspace{0.2ex}^{#1}}}}
\ProvideDocumentCommand\dv{o m g}{%
  \ensuremath{%
    \IfValueTF{#3}{%
      \IfNoValueTF{#1}{%
        \frac{\dd #2}{\dd #3}%
      }{%
        \frac{\dd^{#1} #2}{\dd #3^{#1}}%
      }%
    }{%
      \IfNoValueTF{#1}{%
        \frac{\dd}{\dd #2}%
      }{%
        \frac{\dd^{#1}}{\dd #2^{#1}}%
      }%
    }%
  }%
}
\providecommand*{\pdv}[3][]{\frac{\partial^{#1}#2}{\partial#3^{#1}}}
%  - others
\DeclareMathOperator{\Lap}{\mathcal{L}}
\DeclareMathOperator{\Var}{Var} % varience
\DeclareMathOperator{\Cov}{Cov} % covarience
\DeclareMathOperator{\E}{E} % expected

% Since the amsthm package isn't loaded

% I prefer the slanted \leq
\let\oldleq\leq % save them in case they're every wanted
\let\oldgeq\geq
\renewcommand{\leq}{\leqslant}
\renewcommand{\geq}{\geqslant}

% % redefine matrix env to allow for alignment, use r as default
% \renewcommand*\env@matrix[1][r]{\hskip -\arraycolsep
%     \let\@ifnextchar\new@ifnextchar
%     \array{*\c@MaxMatrixCols #1}}


%\usepackage{framed}
%\usepackage{titletoc}
%\usepackage{etoolbox}
%\usepackage{lmodern}


%\patchcmd{\tableofcontents}{\contentsname}{\sffamily\contentsname}{}{}

%\renewenvironment{leftbar}
%{\def\FrameCommand{\hspace{6em}%
%		{\color{myyellow}\vrule width 2pt depth 6pt}\hspace{1em}}%
%	\MakeFramed{\parshape 1 0cm \dimexpr\textwidth-6em\relax\FrameRestore}\vskip2pt%
%}
%{\endMakeFramed}

%\titlecontents{chapter}
%[0em]{\vspace*{2\baselineskip}}
%{\parbox{4.5em}{%
%		\hfill\Huge\sffamily\bfseries\color{myred}\thecontentspage}%
%	\vspace*{-2.3\baselineskip}\leftbar\textsc{\small\chaptername~\thecontentslabel}\\\sffamily}
%{}{\endleftbar}
%\titlecontents{section}
%[8.4em]
%{\sffamily\contentslabel{3em}}{}{}
%{\hspace{0.5em}\nobreak\itshape\color{myred}\contentspage}
%\titlecontents{subsection}
%[8.4em]
%{\sffamily\contentslabel{3em}}{}{}  
%{\hspace{0.5em}\nobreak\itshape\color{myred}\contentspage}



%%%%%%%%%%%%%%%%%%%%%%%%%%%%%%%%%%%%%%%%%%%
% TABLE OF CONTENTS
%%%%%%%%%%%%%%%%%%%%%%%%%%%%%%%%%%%%%%%%%%%

\usepackage{tikz}
\definecolor{doc}{RGB}{0,60,110}
\usepackage{titletoc}
\contentsmargin{0cm}
\titlecontents{chapter}[3.7pc]
{\addvspace{30pt}%
	\begin{tikzpicture}[remember picture, overlay]%
		\draw[fill=doc!60,draw=doc!60] (-7,-.1) rectangle (-0.9,.5);%
		\pgftext[left,x=-3.5cm,y=0.2cm]{\color{white}\Large\sc\bfseries Chapter\ \thecontentslabel};%
	\end{tikzpicture}\color{doc!60}\large\sc\bfseries}%
{}
{}
{\;\titlerule\;\large\sc\bfseries Page \thecontentspage
	\begin{tikzpicture}[remember picture, overlay]
		\draw[fill=doc!60,draw=doc!60] (2pt,0) rectangle (4,0.1pt);
	\end{tikzpicture}}%
\titlecontents{section}[3.7pc]
{\addvspace{2pt}}
{\contentslabel[\thecontentslabel]{2pc}}
{}
{\hfill\small \thecontentspage}
[]
\titlecontents*{subsection}[3.7pc]
{\addvspace{-1pt}\small}
{}
{}
{\ --- \small\thecontentspage}
[ \textbullet\ ][]

\makeatletter
\renewcommand{\tableofcontents}{%
	\chapter*{%
	  \vspace*{-20\p@}%
	  \begin{tikzpicture}[remember picture, overlay]%
		  \pgftext[right,x=15cm,y=0.2cm]{\color{doc!60}\Huge\sc\bfseries \contentsname};%
		  \draw[fill=doc!60,draw=doc!60] (13,-.75) rectangle (20,1);%
		  \clip (13,-.75) rectangle (20,1);
		  \pgftext[right,x=15cm,y=0.2cm]{\color{white}\Huge\sc\bfseries \contentsname};%
	  \end{tikzpicture}}%
	\@starttoc{toc}}
\makeatother


%From M275 "Topology" at SJSU
\newcommand{\id}{\mathrm{id}}
\newcommand{\taking}[1]{\xrightarrow{#1}}
\newcommand{\inv}{^{-1}}

%From M170 "Introduction to Graph Theory" at SJSU
\DeclareMathOperator{\diam}{diam}
\DeclareMathOperator{\ord}{ord}
\newcommand{\defeq}{\overset{\mathrm{def}}{=}}

%From the USAMO .tex files
\newcommand{\ts}{\textsuperscript}
\newcommand{\dg}{^\circ}
\newcommand{\ii}{\item}

% % From Math 55 and Math 145 at Harvard
% \newenvironment{subproof}[1][Proof]{%
% \begin{proof}[#1] \renewcommand{\qedsymbol}{$\blacksquare$}}%
% {\end{proof}}

\newcommand{\liff}{\leftrightarrow}
\newcommand{\lthen}{\rightarrow}
\newcommand{\opname}{\operatorname}
\newcommand{\surjto}{\twoheadrightarrow}
\newcommand{\injto}{\hookrightarrow}
\newcommand{\On}{\mathrm{On}} % ordinals
\DeclareMathOperator{\img}{im} % Image
\DeclareMathOperator{\Img}{Im} % Image
\DeclareMathOperator{\coker}{coker} % Cokernel
\DeclareMathOperator{\Coker}{Coker} % Cokernel
\DeclareMathOperator{\Ker}{Ker} % Kernel
\DeclareMathOperator{\rank}{rank}
\DeclareMathOperator{\Spec}{Spec} % spectrum
\DeclareMathOperator{\Tr}{Tr} % trace
\DeclareMathOperator{\pr}{pr} % projection
\DeclareMathOperator{\ext}{ext} % extension
\DeclareMathOperator{\pred}{pred} % predecessor
\DeclareMathOperator{\dom}{dom} % domain
\DeclareMathOperator{\ran}{ran} % range
\DeclareMathOperator{\Hom}{Hom} % homomorphism
\DeclareMathOperator{\Mor}{Mor} % morphisms
\DeclareMathOperator{\End}{End} % endomorphism

\newcommand{\eps}{\epsilon}
\newcommand{\veps}{\varepsilon}
\newcommand{\ol}{\overline}
\newcommand{\ul}{\underline}
\newcommand{\wt}{\widetilde}
\newcommand{\wh}{\widehat}
\newcommand{\vocab}[1]{\textbf{\color{blue} #1}}
\providecommand{\half}{\frac{1}{2}}
\newcommand{\dang}{\measuredangle} %% Directed angle
\newcommand{\ray}[1]{\overrightarrow{#1}}
\newcommand{\seg}[1]{\overline{#1}}
\newcommand{\arc}[1]{\wideparen{#1}}
\DeclareMathOperator{\cis}{cis}
\DeclareMathOperator*{\lcm}{lcm}
\DeclareMathOperator*{\argmin}{arg min}
\DeclareMathOperator*{\argmax}{arg max}
\newcommand{\cycsum}{\sum_{\mathrm{cyc}}}
\newcommand{\symsum}{\sum_{\mathrm{sym}}}
\newcommand{\cycprod}{\prod_{\mathrm{cyc}}}
\newcommand{\symprod}{\prod_{\mathrm{sym}}}
\newcommand{\Qed}{\begin{flushright}\qed\end{flushright}}
\newcommand{\parinn}{\setlength{\parindent}{1cm}}
\newcommand{\parinf}{\setlength{\parindent}{0cm}}
% \newcommand{\norm}{\|\cdot\|}
\newcommand{\inorm}{\norm_{\infty}}
\newcommand{\opensets}{\{V_{\alpha}\}_{\alpha\in I}}
\newcommand{\oset}{V_{\alpha}}
\newcommand{\opset}[1]{V_{\alpha_{#1}}}
\newcommand{\lub}{\text{lub}}
\newcommand{\del}[2]{\frac{\partial #1}{\partial #2}}
\newcommand{\Del}[3]{\frac{\partial^{#1} #2}{\partial^{#1} #3}}
\newcommand{\deld}[2]{\dfrac{\partial #1}{\partial #2}}
\newcommand{\Deld}[3]{\dfrac{\partial^{#1} #2}{\partial^{#1} #3}}
\newcommand{\lm}{\lambda}
\newcommand{\uin}{\mathbin{\rotatebox[origin=c]{90}{$\in$}}}
\newcommand{\usubset}{\mathbin{\rotatebox[origin=c]{90}{$\subset$}}}
\newcommand{\lt}{\left}
\newcommand{\rt}{\right}
\newcommand{\bs}[1]{\boldsymbol{#1}}
\newcommand{\exs}{\exists}
\newcommand{\st}{\strut}
\newcommand{\dps}[1]{\displaystyle{#1}}

\newcommand{\sol}{\setlength{\parindent}{0cm}\textbf{\textit{Solution:}}\setlength{\parindent}{1cm} }
\newcommand{\solve}[1]{\setlength{\parindent}{0cm}\textbf{\textit{Solution: }}\setlength{\parindent}{1cm}#1 \Qed}

% Things Lie
\newcommand{\kb}{\mathfrak b}
\newcommand{\kg}{\mathfrak g}
\newcommand{\kh}{\mathfrak h}
\newcommand{\kn}{\mathfrak n}
\newcommand{\ku}{\mathfrak u}
\newcommand{\kz}{\mathfrak z}
\DeclareMathOperator{\Ext}{Ext} % Ext functor
\DeclareMathOperator{\Tor}{Tor} % Tor functor
\newcommand{\gl}{\opname{\mathfrak{gl}}} % frak gl group
\renewcommand{\sl}{\opname{\mathfrak{sl}}} % frak sl group chktex 6

% More script letters etc.
\newcommand{\SA}{\mathcal A}
\newcommand{\SB}{\mathcal B}
\newcommand{\SC}{\mathcal C}
\newcommand{\SF}{\mathcal F}
\newcommand{\SG}{\mathcal G}
\newcommand{\SH}{\mathcal H}
\newcommand{\OO}{\mathcal O}

\newcommand{\SCA}{\mathscr A}
\newcommand{\SCB}{\mathscr B}
\newcommand{\SCC}{\mathscr C}
\newcommand{\SCD}{\mathscr D}
\newcommand{\SCE}{\mathscr E}
\newcommand{\SCF}{\mathscr F}
\newcommand{\SCG}{\mathscr G}
\newcommand{\SCH}{\mathscr H}

% Mathfrak primes
\newcommand{\km}{\mathfrak m}
\newcommand{\kp}{\mathfrak p}
\newcommand{\kq}{\mathfrak q}

% number sets
\newcommand{\RR}[1][]{\ensuremath{\ifstrempty{#1}{\mathbb{R}}{\mathbb{R}^{#1}}}}
\newcommand{\NN}[1][]{\ensuremath{\ifstrempty{#1}{\mathbb{N}}{\mathbb{N}^{#1}}}}
\newcommand{\ZZ}[1][]{\ensuremath{\ifstrempty{#1}{\mathbb{Z}}{\mathbb{Z}^{#1}}}}
\newcommand{\QQ}[1][]{\ensuremath{\ifstrempty{#1}{\mathbb{Q}}{\mathbb{Q}^{#1}}}}
\newcommand{\CC}[1][]{\ensuremath{\ifstrempty{#1}{\mathbb{C}}{\mathbb{C}^{#1}}}}
\newcommand{\PP}[1][]{\ensuremath{\ifstrempty{#1}{\mathbb{P}}{\mathbb{P}^{#1}}}}
\newcommand{\HH}[1][]{\ensuremath{\ifstrempty{#1}{\mathbb{H}}{\mathbb{H}^{#1}}}}
\newcommand{\FF}[1][]{\ensuremath{\ifstrempty{#1}{\mathbb{F}}{\mathbb{F}^{#1}}}}
% expected value
\newcommand{\EE}{\ensuremath{\mathbb{E}}}
\newcommand{\charin}{\text{ char }}
\DeclareMathOperator{\sign}{sign}
\DeclareMathOperator{\Aut}{Aut}
\DeclareMathOperator{\Inn}{Inn}
\DeclareMathOperator{\Syl}{Syl}
\DeclareMathOperator{\Gal}{Gal}
\DeclareMathOperator{\GL}{GL} % General linear group
\DeclareMathOperator{\SL}{SL} % Special linear group

%---------------------------------------
% BlackBoard Math Fonts :-
%---------------------------------------

%Captital Letters
\newcommand{\bbA}{\mathbb{A}}	\newcommand{\bbB}{\mathbb{B}}
\newcommand{\bbC}{\mathbb{C}}	\newcommand{\bbD}{\mathbb{D}}
\newcommand{\bbE}{\mathbb{E}}	\newcommand{\bbF}{\mathbb{F}}
\newcommand{\bbG}{\mathbb{G}}	\newcommand{\bbH}{\mathbb{H}}
\newcommand{\bbI}{\mathbb{I}}	\newcommand{\bbJ}{\mathbb{J}}
\newcommand{\bbK}{\mathbb{K}}	\newcommand{\bbL}{\mathbb{L}}
\newcommand{\bbM}{\mathbb{M}}	\newcommand{\bbN}{\mathbb{N}}
\newcommand{\bbO}{\mathbb{O}}	\newcommand{\bbP}{\mathbb{P}}
\newcommand{\bbQ}{\mathbb{Q}}	\newcommand{\bbR}{\mathbb{R}}
\newcommand{\bbS}{\mathbb{S}}	\newcommand{\bbT}{\mathbb{T}}
\newcommand{\bbU}{\mathbb{U}}	\newcommand{\bbV}{\mathbb{V}}
\newcommand{\bbW}{\mathbb{W}}	\newcommand{\bbX}{\mathbb{X}}
\newcommand{\bbY}{\mathbb{Y}}	\newcommand{\bbZ}{\mathbb{Z}}

%---------------------------------------
% MathCal Fonts :-
%---------------------------------------

%Captital Letters
\newcommand{\mcA}{\mathcal{A}}	\newcommand{\mcB}{\mathcal{B}}
\newcommand{\mcC}{\mathcal{C}}	\newcommand{\mcD}{\mathcal{D}}
\newcommand{\mcE}{\mathcal{E}}	\newcommand{\mcF}{\mathcal{F}}
\newcommand{\mcG}{\mathcal{G}}	\newcommand{\mcH}{\mathcal{H}}
\newcommand{\mcI}{\mathcal{I}}	\newcommand{\mcJ}{\mathcal{J}}
\newcommand{\mcK}{\mathcal{K}}	\newcommand{\mcL}{\mathcal{L}}
\newcommand{\mcM}{\mathcal{M}}	\newcommand{\mcN}{\mathcal{N}}
\newcommand{\mcO}{\mathcal{O}}	\newcommand{\mcP}{\mathcal{P}}
\newcommand{\mcQ}{\mathcal{Q}}	\newcommand{\mcR}{\mathcal{R}}
\newcommand{\mcS}{\mathcal{S}}	\newcommand{\mcT}{\mathcal{T}}
\newcommand{\mcU}{\mathcal{U}}	\newcommand{\mcV}{\mathcal{V}}
\newcommand{\mcW}{\mathcal{W}}	\newcommand{\mcX}{\mathcal{X}}
\newcommand{\mcY}{\mathcal{Y}}	\newcommand{\mcZ}{\mathcal{Z}}


%---------------------------------------
% Bold Math Fonts :-
%---------------------------------------

%Captital Letters
\newcommand{\bmA}{\boldsymbol{A}}	\newcommand{\bmB}{\boldsymbol{B}}
\newcommand{\bmC}{\boldsymbol{C}}	\newcommand{\bmD}{\boldsymbol{D}}
\newcommand{\bmE}{\boldsymbol{E}}	\newcommand{\bmF}{\boldsymbol{F}}
\newcommand{\bmG}{\boldsymbol{G}}	\newcommand{\bmH}{\boldsymbol{H}}
\newcommand{\bmI}{\boldsymbol{I}}	\newcommand{\bmJ}{\boldsymbol{J}}
\newcommand{\bmK}{\boldsymbol{K}}	\newcommand{\bmL}{\boldsymbol{L}}
\newcommand{\bmM}{\boldsymbol{M}}	\newcommand{\bmN}{\boldsymbol{N}}
\newcommand{\bmO}{\boldsymbol{O}}	\newcommand{\bmP}{\boldsymbol{P}}
\newcommand{\bmQ}{\boldsymbol{Q}}	\newcommand{\bmR}{\boldsymbol{R}}
\newcommand{\bmS}{\boldsymbol{S}}	\newcommand{\bmT}{\boldsymbol{T}}
\newcommand{\bmU}{\boldsymbol{U}}	\newcommand{\bmV}{\boldsymbol{V}}
\newcommand{\bmW}{\boldsymbol{W}}	\newcommand{\bmX}{\boldsymbol{X}}
\newcommand{\bmY}{\boldsymbol{Y}}	\newcommand{\bmZ}{\boldsymbol{Z}}
%Small Letters
\newcommand{\bma}{\boldsymbol{a}}	\newcommand{\bmb}{\boldsymbol{b}}
\newcommand{\bmc}{\boldsymbol{c}}	\newcommand{\bmd}{\boldsymbol{d}}
\newcommand{\bme}{\boldsymbol{e}}	\newcommand{\bmf}{\boldsymbol{f}}
\newcommand{\bmg}{\boldsymbol{g}}	\newcommand{\bmh}{\boldsymbol{h}}
\newcommand{\bmi}{\boldsymbol{i}}	\newcommand{\bmj}{\boldsymbol{j}}
\newcommand{\bmk}{\boldsymbol{k}}	\newcommand{\bml}{\boldsymbol{l}}
\newcommand{\bmm}{\boldsymbol{m}}	\newcommand{\bmn}{\boldsymbol{n}}
\newcommand{\bmo}{\boldsymbol{o}}	\newcommand{\bmp}{\boldsymbol{p}}
\newcommand{\bmq}{\boldsymbol{q}}	\newcommand{\bmr}{\boldsymbol{r}}
\newcommand{\bms}{\boldsymbol{s}}	\newcommand{\bmt}{\boldsymbol{t}}
\newcommand{\bmu}{\boldsymbol{u}}	\newcommand{\bmv}{\boldsymbol{v}}
\newcommand{\bmw}{\boldsymbol{w}}	\newcommand{\bmx}{\boldsymbol{x}}
\newcommand{\bmy}{\boldsymbol{y}}	\newcommand{\bmz}{\boldsymbol{z}}

%---------------------------------------
% Scr Math Fonts :-
%---------------------------------------

\newcommand{\sA}{{\mathscr{A}}}   \newcommand{\sB}{{\mathscr{B}}}
\newcommand{\sC}{{\mathscr{C}}}   \newcommand{\sD}{{\mathscr{D}}}
\newcommand{\sE}{{\mathscr{E}}}   \newcommand{\sF}{{\mathscr{F}}}
\newcommand{\sG}{{\mathscr{G}}}   \newcommand{\sH}{{\mathscr{H}}}
\newcommand{\sI}{{\mathscr{I}}}   \newcommand{\sJ}{{\mathscr{J}}}
\newcommand{\sK}{{\mathscr{K}}}   \newcommand{\sL}{{\mathscr{L}}}
\newcommand{\sM}{{\mathscr{M}}}   \newcommand{\sN}{{\mathscr{N}}}
\newcommand{\sO}{{\mathscr{O}}}   \newcommand{\sP}{{\mathscr{P}}}
\newcommand{\sQ}{{\mathscr{Q}}}   \newcommand{\sR}{{\mathscr{R}}}
\newcommand{\sS}{{\mathscr{S}}}   \newcommand{\sT}{{\mathscr{T}}}
\newcommand{\sU}{{\mathscr{U}}}   \newcommand{\sV}{{\mathscr{V}}}
\newcommand{\sW}{{\mathscr{W}}}   \newcommand{\sX}{{\mathscr{X}}}
\newcommand{\sY}{{\mathscr{Y}}}   \newcommand{\sZ}{{\mathscr{Z}}}


%---------------------------------------
% Math Fraktur Font
%---------------------------------------

%Captital Letters
\newcommand{\mfA}{\mathfrak{A}}	\newcommand{\mfB}{\mathfrak{B}}
\newcommand{\mfC}{\mathfrak{C}}	\newcommand{\mfD}{\mathfrak{D}}
\newcommand{\mfE}{\mathfrak{E}}	\newcommand{\mfF}{\mathfrak{F}}
\newcommand{\mfG}{\mathfrak{G}}	\newcommand{\mfH}{\mathfrak{H}}
\newcommand{\mfI}{\mathfrak{I}}	\newcommand{\mfJ}{\mathfrak{J}}
\newcommand{\mfK}{\mathfrak{K}}	\newcommand{\mfL}{\mathfrak{L}}
\newcommand{\mfM}{\mathfrak{M}}	\newcommand{\mfN}{\mathfrak{N}}
\newcommand{\mfO}{\mathfrak{O}}	\newcommand{\mfP}{\mathfrak{P}}
\newcommand{\mfQ}{\mathfrak{Q}}	\newcommand{\mfR}{\mathfrak{R}}
\newcommand{\mfS}{\mathfrak{S}}	\newcommand{\mfT}{\mathfrak{T}}
\newcommand{\mfU}{\mathfrak{U}}	\newcommand{\mfV}{\mathfrak{V}}
\newcommand{\mfW}{\mathfrak{W}}	\newcommand{\mfX}{\mathfrak{X}}
\newcommand{\mfY}{\mathfrak{Y}}	\newcommand{\mfZ}{\mathfrak{Z}}
%Small Letters
\newcommand{\mfa}{\mathfrak{a}}	\newcommand{\mfb}{\mathfrak{b}}
\newcommand{\mfc}{\mathfrak{c}}	\newcommand{\mfd}{\mathfrak{d}}
\newcommand{\mfe}{\mathfrak{e}}	\newcommand{\mff}{\mathfrak{f}}
\newcommand{\mfg}{\mathfrak{g}}	\newcommand{\mfh}{\mathfrak{h}}
\newcommand{\mfi}{\mathfrak{i}}	\newcommand{\mfj}{\mathfrak{j}}
\newcommand{\mfk}{\mathfrak{k}}	\newcommand{\mfl}{\mathfrak{l}}
\newcommand{\mfm}{\mathfrak{m}}	\newcommand{\mfn}{\mathfrak{n}}
\newcommand{\mfo}{\mathfrak{o}}	\newcommand{\mfp}{\mathfrak{p}}
\newcommand{\mfq}{\mathfrak{q}}	\newcommand{\mfr}{\mathfrak{r}}
\newcommand{\mfs}{\mathfrak{s}}	\newcommand{\mft}{\mathfrak{t}}
\newcommand{\mfu}{\mathfrak{u}}	\newcommand{\mfv}{\mathfrak{v}}
\newcommand{\mfw}{\mathfrak{w}}	\newcommand{\mfx}{\mathfrak{x}}
\newcommand{\mfy}{\mathfrak{y}}	\newcommand{\mfz}{\mathfrak{z}}


\usepackage{mathtools}

\title{\Huge{Discrete Mathematics}\\Week 8}
\author{\huge{Abeyah Calpatura}}
\date{}

\begin{document}
\maketitle
\section*{6.1}
\subsection*{Exercises} \\
\text{Abeyah Calpatura} \\
\#4ab, 7ab, 8, 9, 12, 15, 20, 31 \\

\noindent \textbf{\#4} \\
\noindent Let A = $\{n \in \mathbb{Z} \mid n = 5r \: \text{for some integer r}\}$ \\
Let B = $\{m \in \mathbb{Z} \mid m = 20s \: \text{for some integer s}\}$ \\
$A = \{\dots, -40, -35, -30, -25, -20, -15, -10, -5, 0, 5, 10, 15, 20, 25, 30, 35, 40, \dots \}$ \\
$B = \{\dots, -100, -80, -60, -40, -20, 0, 20, 40, 60, 80, 100, \dots \}$ \\ \\
\textbf{\#4a} 
\sol{ $A \subseteq B$
    \begin{align*}
    & \text{Not true, prove that we need to find at least one element that belongs to A but not B.} \\
    & \text{Let's take the element 5.} \\
    & \text{5 belongs to A because 5 = 5(1).} \\
    & \text{5 does not belong to B because 5 $\neq$ 20s for any integer s.} \\
    & \text{Therefore, A $\nsubseteq$ B.} \\
    \end{align*}
}
\textbf{\#4b} 
\sol{ $B \subseteq A$
    \begin{align*}
    & \text{A is the set of all numbers divisble by 5.} \\
    & \text{B, have all integers that are divisble by 20. } \\
    & \text{$z \in B$ must be divisble by 4 and 5} \\
    & \text{Therefore, B $\subseteq$ A. Every element of B also belongs to A} \\
    \end{align*}
} 
\textbf{\#7} \\
\noindent Let A = $\{x \in \mathbb{Z} \mid x = 6a + 4 \: \text{for some integer a}\}$ \\
Let B = $\{y \in \mathbb{Z} \mid y = 18b - 2 \: \text{for some integer b}\}$ \\
Let C = $\{z \in \mathbb{Z} \mid z = 18c + 16 \: \text{for some integer c}\}$ \\
Prove or disprove each of the following statements. \\ \\
\textbf{\#7a}
\sol{ $A \subseteq B$
    \begin{align*}
    & \text{For a = 1, we have that $x=6\cdot 1 + 4 = 10 \in A$} \\
    & \text{Suppose that x = 10. Solve the eqution 10 = 18b - 2} \\
    & \text{12 = 18b} \\
    & \text{2 = 3b} \\
    & \text{$b = \frac{2}{3} \notin \mathbb{Z}$} \\
    & \text{Therefore, A $\nsubseteq$ B.} \\
    \end{align*}
}
\textbf{\#7b}
\sol{ $B \subseteq A$
    \begin{align*}
    & \text{A is the set of all numbers that are 4 more than a multiple of 6.} \\
    & \text{B is the set of all numbers that are 3 less than a multiple of 10.} \\
    & \text{There is an integer b so that $x = 18b-2$} \\
    & \text{$x= 18b - 2= 6a +4$} \\
    & \text{$18b - 2 - 4 = 6a$} \\
    & \text{$18b - 6 = 6a$} \\
    & \text{$6(3b - 1) = 6a$} \\
    & \text{$3b - 1 = a$} \\
    & \text{Therefore, B $\subseteq$ A. Every element of B also belongs to A.} \\
    \end{align*}
}
\textbf{\#8a}
\sol{ $\{x \in U \mid x \in A \: \text{and} \: x \in B\}$
    \begin{align*}
    & \text{This is the set of all x from U, such that x is both in A and in B.} \\
    & \text{Solution: $A \cap B$} \\
    \end{align*}
}
\textbf{\#8b}
\sol{ $\{x \in U \mid x \in A \: \text{or} \: x \in B\}$
    \begin{align*}
    & \text{This is the set of all x from U, such that x is in A or in B.} \\
    & \text{Solution: $A \cup B$} \\
    \end{align*}
}
\textbf{\#8c} 
\sol{ $\{x \in U \mid x \in A \: \text{and} \: x \notin B\}$
    \begin{align*}
    & \text{This is the set of all x from U, such that x is in A and not in B.} \\
    & \text{Solution: $A - B$} \\
    \end{align*}
}
\textbf{\#8d} 
\sol{ $\{x \in U \mid x \notin A\}$
    \begin{align*}
    & \text{This is the set of all x from U that do not belong to A} \\
    & \text{Solution: $A^c$} \\
    \end{align*}
}
\textbf{\#9a}
\sol{ $x \notin A \cup B$ if, and only if, $x \notin A$ and $x \notin B$.} \\
\textbf{\#9b}
\sol{ $x \notin A \cap B$ if, and only if, $x \notin A$ or $x \notin B$.} \\
\textbf{\#9c}
\sol{ $x \in A - B$ if, and only if, $x \notin A$ and $x \in B$.} \\
\\
\textbf{\#12} \\
Let the universal set be $\mathbb{R}$, the set of all real numbers, and let A = $\{x \in \mathbb{R} \mid -3 \leq x \leq 0\}$ and B = $\{x \in \mathbb{R} \mid -1 < x < 2\}$, and C = $\{x \in \mathbb{R} \mid 6 < x \leq 8\}$ \\
\textbf{\#12a}
\sol{ $A \cup B$
    \begin{align*}
    & \text{$\{x \in \mathbb{R} \mid -3 \leq x \leq 0 \: \text{or} \: -1 < x < 2\} = \{x \in \mathbb{R} \mid -3 \leq x < 2\}$} \\
    \end{align*}
}
\textbf{\#12b}
\sol{ $A \cap B$
    \begin{align*}
    & \text{$\{x \in \mathbb{R} \mid -3 \leq x \leq 0 \: \text{and} \: -1 < x < 2\} = \{x \in \mathbb{R} \mid -1 < x \leq 0\}$} \\
    \end{align*}
}
\textbf{\#12c}
\sol{ $A^c$
    \begin{align*}
    & \text{$\{x \in \mathbb{R} \mid -3 \leq x \leq 0\}^c = \{x \in \mathbb{R} \mid x < -3 \: \text{or} \: x > 0\}$} \\
    \end{align*}
}
\textbf{\#12d}
\sol{ $A \cup C$
    \begin{align*}
    & \text{$\{x \in \mathbb{R} \mid -3 \leq x \leq 0 \: \text{or} \: 6 < x \leq 8\} = \{x \in \mathbb{R} \mid -3 \leq x \leq 8\}$} \\
    \end{align*}
}
\textbf{\#12e}
\sol{ $A \cap C$
    \begin{align*}
    & \text{$\{x \in \mathbb{R} \mid -3 \leq x \leq 0 \: \text{and} \: 6 < x \leq 8\} = \emptyset$} \\
    \end{align*}
}
\textbf{\#12f}
\sol{ $B^c$
    \begin{align*}
    & \text{$\{x \in \mathbb{R} \mid -1 < x < 2\}^c = \{x \in \mathbb{R} \mid x \leq -1 \: \text{or} \: x \geq 2\}$} \\
    \end{align*}
}
\textbf{\#12g}
\sol{ $A^c \cap B^c$
    \begin{align*}
    & \text{$\{x \in \mathbb{R} \mid x < -3 \: \text{or} \: x > 0\} \: \text{and} \: \{x \in \mathbb{R} \mid x \leq -1 \: \text{or} \: x \geq 2\} = \{x \in \mathbb{R} \mid x < -3 \: \text{or} \: x \geq 2\}$} \\
    \end{align*}
}
\textbf{\#12h}
\sol{ $A^c \cup B^c$
    \begin{align*}
    & \text{$\{x \in \mathbb{R} \mid x < -3 \: \text{or} \: x > 0\} \: \text{or} \: \{x \in \mathbb{R} \mid x \leq -1 \: \text{or} \: x \geq 2\} = \{x \in \mathbb{R} \mid x < -3 \: \text{or} \: x \geq 2\}$} \\
    \end{align*}
}
\textbf{\#12i}
\sol{ $(A \cap B)^c$
    \begin{align*}
    & \text{$\{x \in \mathbb{R} \mid -1 < x \leq 0\}^c = \{x \in \mathbb{R} \mid x \leq -1 \: \text{or} \: x > 0\}$} \\
    \end{align*}
}
\textbf{\#12j}
\sol{ $(A \cup B)^c$
    \begin{align*}
    & \text{$\{x \in \mathbb{R} \mid -3 \leq x < 2\}^c = \{x \in \mathbb{R} \mid x < -3 \: \text{or} \: x \geq 2\}$} \\
    \end{align*}
}
\textbf{\#15} \\
Venn Diagram \\
\newpage
\noindent \textbf{\#20} \\
Let $B_i = \{x \in \mathbb{R} \mid 0 \leq x \leq i\}$ for each integer i = 1, 2, 3, 4 \\
\textbf{\#20a}
\sol{ $B_1 \cup B_2 \cup B_3 \cup B_4 = $
    \begin{align*}
    & \text{$\{x \in \mathbb{R} \mid 0 \leq x \leq 1\} \cup \{x \in \mathbb{R} \mid 0 \leq x \leq 2\} \cup \{x \in \mathbb{R} \mid 0 \leq x \leq 3\} \cup \{x \in \mathbb{R} \mid 0 \leq x \leq 4\}$} \\
    & \text{$= \{x \in \mathbb{R} \mid 0 \leq x \leq 4\}$} \\
    \end{align*}
}
\textbf{\#20b}
\sol{ $B_1 \cap B_2 \cap B_3 \cap B_4 = $
    \begin{align*}
    & \text{$\{x \in \mathbb{R} \mid 0 \leq x \leq 1\} \cap \{x \in \mathbb{R} \mid 0 \leq x \leq 2\} \cap \{x \in \mathbb{R} \mid 0 \leq x \leq 3\} \cap \{x \in \mathbb{R} \mid 0 \leq x \leq 4\}$} \\
    & \text{$= \{x \in \mathbb{R} \mid 0 \leq x \leq 1\}$} \\
    \end{align*}
}
\textbf{\#20c}
\sol{ Are $B_1, B_2, B_3, and B_4$ mutually disjoint? \\
    \begin{align*}
    & \text{No, they are not mutually disjoint.} \\
    & \text{The intersection of any two of them is not empty.} \\
    \end{align*}
}
\textbf{\#31} \\
Suppose $A = \{1, 2\}$ and $B = \{2, 3\}$. Find each of the following \\
\textbf{\#31a}
\sol{ $\mathcal{P}(A \cap B)$
    \begin{align*}
    & \text{$A \cap B = \{2\}$} \\
    & \text{$\mathcal{P}(A \cap B) = \{\emptyset, \{2\}\}$} \\
    \end{align*}
}
\textbf{\#31b}
\sol{ $\mathcal{P}(A)$
    \begin{align*}
    & \text{$\mathcal{P}(A) = \{\emptyset, \{1\}, \{2\}, \{1, 2\}\}$} \\
    \end{align*}
}
\textbf{\#31c}
\sol{ $\mathcal{P}(A \cup B)$
    \begin{align*}
    & \text{$A \cup B = \{1, 2, 3\}$} \\
    & \text{$\mathcal{P}(A \cup B) = \{\emptyset, \{1\}, \{2\}, \{3\}, \{1, 2\}, \{1, 3\}, \{2, 3\}, \{1, 2, 3\}\}$} \\
    \end{align*}
}
\textbf{\#31d}
\sol{ $\mathcal{P}(A \times B)$
    \begin{align*}
    & \text{$A \times B = \{(1, 2), (1, 3), (2, 2), (2, 3)\}$} \\
    & \text{$\mathcal{P}(A \times B) = \{\emptyset, \{(1, 2)\}, \{(1, 3)\}, \{(2, 2)\}, \{(2, 3)\}, \{(1, 2), (1, 3)\}, \{(1, 2), (2, 2)\}$} \\ 
    & \text{$\{(1, 2), (2, 3)\}, \{(1, 3), (2, 2)\}, \{(1, 3), (2, 3)\}, \{(2, 2), (2, 3)\}, \{(1, 2), (1, 3), (2, 2)\}, \{(1, 2), (1, 3)$} \\
    & \text{$, (2, 3)\}, \{(1, 2), (2, 2), (2, 3)\}, \{(1, 3), (2, 2), (2, 3)\}, \{(1, 2), (1, 3), (2, 2), (2, 3)\}$} \\
    \end{align*}
}}
\newpage
\section*{7.1}
\subsection*{Exercises} \\
\text{Abeyah Calpatura} \\
\#1, 2, 8cd, 11cd, 12cd, 17, 18 \\
\textbf{\#1} \\
Let X = $\{1, 3, 5\}$ and Y = $\{s, t, u, v\}$. Define f: $X \rightarrow Y$ \\
\vspace{3cm} \\
\textbf{\#1a}
\sol{
    \begin{align*}
    & \text{Domain f = $\{1, 3, 5\}$} \\
    & \text{Codomain f = $\{s, t, u, v\}$} \\
    \end{align*}
}
\textbf{\#1b}
\sol{ Find f(1), f(3), f(5)
    \begin{align*}
    & \text{f(1) = v} \\
    & \text{f(3) = s} \\
    & \text{f(5) = v} \\
    \end{align*}
}
\textbf{\#1c}
\sol{ What is the range of f?
    \begin{align*}
    & \text{Range f = $\{s, v\}$} \\
    \end{align*}
}
\textbf{\#1d} 
\sol{ Is 3 an inverse image of s? Is 1 an inverse image of u?
    \begin{align*}
    & \text{3 is an inverse image of s.} \\
    & \text{1 is not an inverse image of u.} \\
    \end{align*}
}
\textbf{\#1e}
\sol{ What is the inverse image of s? of u? of v?
    \begin{align*}
    & \text{Inverse image of s = $\{3\}$} \\
    & \text{Inverse image of u = $\emptyset$} \\
    & \text{Inverse image of v = $\{1, 5\}$} \\
    \end{align*}
}
\textbf{\#1f} 
\sol{ Represent f as a set of ordered pairs. 
    \begin{align*}
    & \text{f = $\{(1, v), (3, s), (5, v)\}$} \\
    \end{align*}
}
\newpage
\noindent \textbf{\#2} \\
Let X = $\{1, 3, 5\}$ and Y = $\{a, b, c, d\}$. Define g: $X \rightarrow Y$ \\
\vspace{3cm} \\
\textbf{\#2a}
\sol{
    \begin{align*}
    & \text{Domain g = $\{1, 3, 5\}$} \\
    & \text{Codomain g = $\{a, b, c, d\}$} \\
    \end{align*}
}
\textbf{\#2b}
\sol{ Find g(1), g(3), g(5)
    \begin{align*}
    & \text{g(1) = b} \\
    & \text{g(3) = b} \\
    & \text{g(5) = b} \\
    \end{align*}
}
\textbf{\#2c}
\sol{ What is the range of g?
    \begin{align*}
    & \text{Range g = $\{b\}$} \\
    \end{align*}
}
\textbf{\#2d}
\sol{ Is 3 an inverse image of a? Is 1 an inverse image of c?
    \begin{align*}
    & \text{3 is not an inverse image of a.} \\
    & \text{1 is the inverse image of c.} \\
    \end{align*}
}
\textbf{\#2e}
\sol{ What is the inverse image of b? of c? 
    \begin{align*}
    & \text{Inverse image of b = $\{1, 3, 5\}$} \\
    & \text{Inverse image of c = $\{\emptyset\}$} \\
    \end{align*}
}
\textbf{\#2f}
\sol{ Represent g as a set of ordered pairs. 
    \begin{align*}
    & \text{g = $\{(1, b), (3, b), (5, b)\}$} \\
    \end{align*}
}
\textbf{\#8} 
Let $J_5 = \{0, 1, 2, 3, 4\}$, and define a function $F: J_5 \rightarrow J_5$ as follows: For each $x \in J_5, F(x) = (x^3+2x+4)$ mod 5. \\
\textbf{\#8c}
\sol{ F(2)
    \begin{align*}
    & \text{F(2) = $(2^3 + 2(2) + 4)$ mod 5} \\
    & \text{F(2) = 16 mod 5} \\
    & \text{F(2) = 1} \\
    \end{align*}
}
\textbf{\#8d}
\sol{ F(3)
    \begin{align*}
    & \text{F(3) = $(3^3 + 2(3) + 4)$ mod 5} \\
    & \text{F(3) = 37 mod 5} \\
    & \text{F(3) = 2} \\
    \end{align*}
}
\textbf{\#11} 
Let $F: \mathbb{Z} \times \mathbb{Z} \rightarrow \mathbb{Z} \times \mathbb{Z}$ as follows: For every ordered pair (a, b) of integers, $F(a, b) = (2a+1, 3b-2)$. \\
\textbf{\#11c}
\sol{ F(3, 2)
    \begin{align*}
    & \text{F(3, 2) = $(2(3) + 1, 3(2) - 2)$} \\
    & \text{F(3, 2) = (7, 4)} \\
    \end{align*}
}
\textbf{\#11d}
\sol{ F(1, 5)
    \begin{align*}
    & \text{F(1, 5) = $(2(1) + 1, 3(5) - 2)$} \\
    & \text{F(1, 5) = (3, 13)} \\
    \end{align*}
}
\textbf{\#12}
Let $J_5={0, 1, 2, 3, 4}$ and define $G: J_5 \times J_5 \rightarrow J_5 \times J_5$ as follows: For each (a, b) $\in J_5 \times J_5$, \\
$G(a, b) = ((2a+1) \:\text{mod}\: 5, (3b-2) \:\text{mod}\: 5)$
\textbf{\#12c}
\sol{ G(3, 2)
    \begin{align*}
    & \text{G(3, 2) = $((2(3) + 1) \:\text{mod}\: 5, (3(2) - 2) \:\text{mod}\: 5)$} \\
    & \text{G(3, 2) = $(7 \:\text{mod}\: 5, 4 \:\text{mod}\: 5)$} \\
    & \text{G(3, 2) = (2, 4)} \\
    \end{align*}
}
\textbf{\#12d}
\sol{ G(1, 5)
    \begin{align*}
    & \text{G(1, 5) = $((2(1) + 1) \:\text{mod}\: 5, (3(5) - 2) \:\text{mod}\: 5)$} \\
    & \text{G(1, 5) = $(3 \:\text{mod}\: 5, 13 \:\text{mod}\: 5)$} \\
    & \text{G(1, 5) = (3, 3)} \\
    \end{align*}
}
\textbf{\#17a}
\sol{ log$_2$ 8
    \begin{align*}
    & \text{log$_2$ 8 = 3} \\
    & \text{2$^3$ = 8} \\
    \end{align*}
}
\textbf{\#17b}
\sol{log$_5(\frac{1}{25})$ = -2
    \begin{align*}
    & \text{log$_5(\frac{1}{25})$ = -2} \\
    & \text{5$^{-2}$ = $\frac{1}{25}$} \\
    \end{align*}
}
\textbf{\#17c}
\sol{ log$_4$4=1
    \begin{align*}
    & \text{log$_4$4 = 1} \\
    & \text{4$^1$ = 4} \\
    \end{align*}
}
\textbf{\#17d}
\sol{ log$_3(3^n)$ = n
    \begin{align*}
    & \text{log$_3(3^n)$ = n} \\
    & \text{3$^n$ = 3$^n$} \\
    \end{align*}    
}
\textbf{\#17e}
\sol{ log$_4$1 = 0
    \begin{align*}
    & \text{log$_4$1 = 0} \\
    & \text{4$^0$ = 1} \\
    \end{align*}
}
\textbf{\#18a}
\sol{log$_3$81
    \begin{align*}
    & \text{log$_3$81 = 4} \\
    & \text{3$^4$ = 81} \\
    \end{align*}
}
\textbf{\#18b}
\sol{log$_2$1024
    \begin{align*}
    & \text{log$_2$1024 = 10} \\
    & \text{2$^{10}$ = 1024} \\
    \end{align*}
}
\textbf{\#18c}
\sol{log$_3$$(\frac{1}{27})$
    \begin{align*}
    & \text{log$_3$$(\frac{1}{27})$ = -3} \\
    & \text{3$^{-3}$ = $\frac{1}{27}$} \\
    \end{align*}
}
\textbf{\#18d}
\sol{log$_2$1
    \begin{align*}
    & \text{log$_2$1 = 0} \\
    & \text{2$^0$ = 1} \\
    \end{align*}
}
\textbf{\#18e}
\sol{log$_10(\frac{1}{10})$
    \begin{align*}
    & \text{log$_10(\frac{1}{10})$ = -1} \\
    & \text{10$^{-1}$ = $\frac{1}{10}$} \\
    \end{align*}
}
\textbf{\#18f}
\sol{log$_2(2^k)$
    \begin{align*}
    & \text{log$_2(2^k)$ = k} \\
    & \text{2$^k$ = 2$^k$} \\
    \end{align*}
}
\newpage
\section*{7.2}
\subsection*{Exercises} \\
\text{Abeyah Calpatura} \\
\#1, 7, 8, 10, 17, 24, 25 \\

\textbf{\#1a} 
\sol{ A funtion F is one-to-one if, and only if, each element in the co-domain of F is the image of at \textbf{most} one element in the domain of F.}
\textbf{\#1b}
\sol{ A function F is onto if, and only if, each element in the co-domain of F is the image of \textbf{at least} one element in the domain of F.}
\textbf{\#7}
Let X = $\{a, b, c, d\}$ and Y = $\{e, f, g\}$. Define functions F and G by arrow diagrams\\
\vspace{6cm} \\
\textbf{\#7a}
\sol{ Is F one-to-one? Why or why not? Is it onto? Why or why not?
    \begin{align*}
    & \text{F is not one-to-one because the element b is the image of both a and c.} \\
    & \text{F is onto because each element in the co-domain is the image of at least one element in the domain.} \\
    \end{align*}
}
\textbf{\#7b}
\sol{ Is G one-to-one? Why or why not? Is it onto? Why or why not?
    \begin{align*}
    & \text{G is one-to-one because each element in the co-domain is the image of at most one element in the domain.} \\
    & \text{G is not onto because the element g is not the image of any element in the domain.} \\
    \end{align*}
}
\textbf{\#8}
Let X = $\{a, b, c\}$ and Y = $\{d, e, f, g\}$. Define functions H and K by arrow diagrams\\
\vspace{6cm} \\
\textbf{\#8a}
\sol{ Is H one-to-one? Why or why not? Is it onto? Why or why not?
    \begin{align*}
    & \text{H is neither one to one nor onto.} \\
    & \text{H(b) = H(c) = y so H is not one to one} \\
    & \text{H never takes the value of x, so it is not onto} \\
    \end{align*}
}
\textbf{\#8b}
\sol{ Is K one-to-one? Why or why not? Is it onto? Why or why not?
    \begin{align*}
    & \text{K is one-to-one but not onto.} \\
    & \text{K takes three different values on the tree elements of X, so it is one-to-one.} \\
    & \text{K never takes the value of z, so it is not onto.} \\
    \end{align*}
}
\textbf{\#10}
\end{document}